\usepackage{geometry}
\geometry{
    a4paper,
    left=30mm,
    right=30mm,
    top=30mm,
    bottom=30mm,
}

\usepackage{placeins}
\usepackage[footnotes,definitionLists,hashEnumerators,smartEllipses,hybrid]{markdown}
\usepackage[utf8]{inputenc}
\usepackage[document]{ragged2e} % remove line justify
\usepackage{dblfloatfix}

\usepackage{microtype}
\usepackage[none]{hyphenat}

\setlength{\RaggedRightParindent}{1.5em}
\setlength{\parskip}{0.6em}

\usepackage{setspace}
\setstretch{1.2} % Optionally adjust line spacing

\setlength{\RaggedRightParindent}{\parindent}
\setlength{\parskip}{1em}

  
\usepackage{fontspec}
%\defaultfontfeatures{Mapping=tex-text,Scale=MatchLowercase}
\setmainfont{Source Sans Pro--Light}[BoldFont={Source Sans Pro-Regular}, ItalicFont={Source Sans Pro-Light Italic},]

\usepackage{xurl} % for '\url' macro
% \usepackage{xcolor}
\usepackage[usenames, dvipsnames, svgnames, table, x11names]{xcolor} % Support for a broad range of colors and additional color models
\definecolor{swissblack}{HTML}{333333}
\definecolor{swissdarkblue}{HTML}{023047}
\definecolor{swissdarkgreen}{HTML}{006666}
\definecolor{swissyellow}{HTML}{FFDC49}
\definecolor{swissred}{HTML}{ff5353}
\definecolor{swisslightblue}{HTML}{53c7ff}
\definecolor{swisslightgreen}{HTML}{50d079}
\definecolor{swisslightorange}{HTML}{ffac67}
\definecolor{darkred}{HTML}{cd0000}

\definecolor{swisslink}{HTML}{005595}%219EBC
% \color{swiss_black} %default

% command to use these colors and formatting; xspace for correct spacing including with punctuation marks.
\usepackage{xspace}
\newcommand{\variablesdarkgreen}[1]{\textbf{\textcolor{swissdarkgreen}{#1}}\xspace}
\newcommand{\variablesdarkred}[1]{\textbf{\textcolor{darkred}{#1}}\xspace}

% \color{swissblack} 

 % green #219EBC
% darkgreen #006666

\usepackage{authblk}
\usepackage{lineno}
\usepackage{enumitem}
\setlist[itemize]{noitemsep}

\usepackage{siunitx} % SI units


\usepackage[colorlinks]{hyperref}
\hypersetup{
    linkcolor={swisslink},
    citecolor={swisslink},
    filecolor=blue!50!black,
    urlcolor=swisslink,
}

\usepackage{natbib}
\setcitestyle{square,numbers,sort&compress}
\usepackage{hypernat}

\usepackage{graphicx}
\graphicspath{ {images/} }

\usepackage{booktabs}
\usepackage{rotating, tabularx}
\usepackage{ltablex}
\usepackage{caption}
\captionsetup{font=normalsize}

\usepackage{pdflscape}
\usepackage{multirow}

\usepackage{fancyhdr}
\pagestyle{fancy}
%\lhead{Lecture 1}
%\rhead{Handout 1}
%\lfoot{Lecture 1}
%\rfoot{Handout 1}
\usepackage{tocloft}  % Customizing the Table of Contents
\setcounter{tocdepth}{1}
\usepackage{pdfpages} % include supplemental pdf
% cover

%%%% Supplemental labels%%%%
%Define command to start a supplemental section
%set the supplemental letter used for figures (e.g. Figure E1)
\newcommand{\beginsupplement}{%
        \setcounter{table}{0}
        \renewcommand{\thetable}{E\arabic{table}}%
        \setcounter{figure}{0}
        \renewcommand{\thefigure}{E\arabic{figure}}%
         }

%\usepackage[a4paper,margin=5pt]{geometry}
% Define a new column type for left-aligned text in a fixed width column
% \newcolumntype{L}[1]{>{\raggedright\arraybackslash}p{#1}}
\newcolumntype{L}[1]{>{\noindent\arraybackslash\hspace{-1.5em}}p{#1}}



%\documentclass[11pt]{article}




\begin{document}
\title{Switzerland Omics - Investor Document}
\newcommand{\IPSNEO}{1}
\author[\IPSNEO]{Dylan Lawless %\thanks{Addresses for correspondence: \href{mailto:Dylan.Lawless@kispi.uzh.ch}{Dylan.Lawless@kispi.uzh.ch}}
}
\affil[\IPSNEO]{Switzerland Omics, Zürich, Switzerland.}

\maketitle
\justify

\section*{Executive summary}
\begin{itemize}[leftmargin=*]
  \item \textbf{Company}: Switzerland Omics AG
  \item \textbf{Location}: Zurich, Switzerland
  \item \textbf{Mission}: To define the future of genomic interpretation through standardised probabilistic frameworks, curated datasets, and intelligent automation.
  \item \textbf{Vision}: Genomics as a persistent and horizontal infrastructure technology—interwoven with every area of human health and biomedical decision-making.
  \item \textbf{Investment ask}: Seeking CHF [insert amount] to expand productisation, scale dataset partnerships, and drive regulatory-ready deployments.
  \item \textbf{Investment thesis}: Genomics is crossing from research to routine care, but clinical interpretation remains bottlenecked by qualitative, non-standardised methods. Switzerland Omics addresses this gap with foundational frameworks and tools already running in real-world diagnostics. This is a deeply technical team working on a category-defining opportunity.
  \item \textbf{Competitive advantage}: Proprietary data assets, novel probabilistic methods, early clinical deployment, and expert-led development.
\end{itemize}

\section*{Problem}
\begin{itemize}[leftmargin=*]
  \item Genomics lacks formal quantitative priors, leading to unquantified uncertainty in variant interpretation.
  \item Clinical and research pipelines remain heuristic, with poor integration of probabilistic outcomes across variant classes.
  \item Absence of standardised tooling or data infrastructure to assess and interpret negative/uncertain signals in addition to positive calls.
\end{itemize}

\section*{Solution}
A suite of products delivering theoretical innovation and applied utility in genomic diagnostics:

\begin{itemize}[leftmargin=*]
  \item \textbf{QV Standard}: A new framework for qualifying variants in clinical pipelines.\\
    \href{https://drive.google.com/file/d/1Hx2woDKequA9k0RkVZfockgFbIDXB6El/view?usp=drivesdk}{[Application Note]} \href{https://switzerlandomics.ch/services/qv_database/}{[Public Resource]} \cite{lawless_application_2025}
  \item \textbf{PanelAppRex}: Unified ontology and API for gene panel logic and curation.\\
    \href{https://drive.google.com/file/d/1gHqq2X7Zzq1s3SQyZ4gr5krbBtqisnfI/view?usp=share_link}{[Application Note]} \href{https://switzerlandomics.ch/services/panelAppRexAi/}{[Public Resource]} \cite{lawless_panelapprex_2025}
  \item \textbf{Quant}: Statistical risk estimator for variant observation probabilities.\\
    \href{https://drive.google.com/file/d/1eerCdctsMi2gRsVxceHGaSMv2ogP0s5W/view?usp=share_link}{[Application Note]} \cite{lawless_quantifying_2025}
  \item \textbf{Bayesian variant interpretation}: 16-state model for causal inference, unifying theoretical and empirical insight.\\
    \href{https://drive.google.com/file/d/1HN_zyZidilCfq8Bkr45d-9suembRnifM/view?usp=sharing}{[Preprint in progress]}
  \item \textbf{GuRu}: ACMG-guided variant curation automation.\\
    \href{https://github.com/DylanLawless/ACMGuru}{[GitHub Repository]}
  \item \textbf{Dante}: End-to-end WGS-to-report system for clinical diagnostics.\\
    \href{https://drive.google.com/file/d/1WHOqRtPvuTukrcRl5Yn6iX7pL-Ye9R07/view?usp=share_link}{[Application Note]}
\end{itemize}

\section*{Market}
\begin{itemize}[leftmargin=*]
  \item \textbf{TAM}: Global genomics and precision medicine market >CHF 100B.
  \item \textbf{Initial focus}: Diagnostics labs, national genomic initiatives, population health platforms.
  \item \textbf{Expansion}: Pharma (biomarker qualification), AI healthcare infrastructure, insurance risk scoring.
\end{itemize}

\section*{Go-to-market strategy}
\begin{itemize}[leftmargin=*]
  \item Clinical diagnostic labs using variant interpretation and reporting tools (GuRu, Dante).
  \item Licensing model for data-driven frameworks (QV Standard, Quant) to academic and government genomics programmes.
  \item Sales led by founder team with academic relationships; future partner-led distribution planned for enterprise.
\end{itemize}

\section*{Regulatory and compliance plan}
\begin{itemize}[leftmargin=*]
  \item Data privacy and security aligned to Swiss and EU requirements (in clinical use already).
  \item CE-IVDR eligibility scoped for report generation (Dante) and variant interpretation (GuRu).
  \item No direct patient contact or therapeutic claim; software classified as clinical support tool.
\end{itemize}

\section*{Traction}
\begin{itemize}[leftmargin=*]
  \item Open-source tools and public datasets actively used in national and research settings.
  \item Hundreds of real clinical cases processed via framework tools (Dante, GuRu).
  \item Letters of intent and institutional interest in place; further documentation available on request.
\end{itemize}

\section*{Team}
\begin{itemize}[leftmargin=*]
  \item \textbf{Founder}: Dylan Lawless – background in statistical genomics, clinical interpretation, and AI infrastructure.
  \item \textbf{Advisory network}: Genetics, bioinformatics, and regulatory experts from academic and clinical environments.
  \item Team roadmap includes first hires in engineering, compliance, and partner success.
\end{itemize}

\section*{Roadmap}
\begin{itemize}[leftmargin=*]
  \item \textbf{H1 2025}: Expand pilot programmes in clinical genomics (GuRu, Dante).
  \item \textbf{H2 2025}: Launch enterprise-ready data packages (Quant, PanelAppRex).
  \item \textbf{2026}: CE-mark regulatory filing; scale commercialisation across EU and UK.
\end{itemize}

\section*{Funding and use of proceeds}
\begin{itemize}[leftmargin=*]
  \item Raising CHF [insert] seed round to fund 18–24 month runway.
  \item \textbf{Use of proceeds}:
    \begin{itemize}
      \item Product development and engineering: 40\%
      \item Regulatory and compliance: 20\%
      \item Business development and partnerships: 25\%
      \item Operational and legal: 15\%
    \end{itemize}
\end{itemize}

\section*{Risks and mitigation}
\begin{itemize}[leftmargin=*]
  \item \textbf{Adoption risk}: mitigated by early clinical deployment and co-development with labs.
  \item \textbf{Regulatory delay}: initial use cases are decision-support and do not require direct certification.
  \item \textbf{Technical complexity}: codebases are modular, published, and in use; risk managed via strong engineering standards.
\end{itemize}

\section*{Financials (summary)}
\begin{itemize}[leftmargin=*]
  \item Currently pre-revenue. Tools are deployed in live clinical diagnostic workflows supporting variant interpretation for hundreds of patients.
  \item Projected CHF 1.5M \ac{ARR} within 24 months based on expected contracts with labs and institutional partners.
  \item Minimal burn rate with founder-led development and clinical collaborations in place (estimated CHF 8K–12K/month).
  \item Bootstrapped to date; no prior external funding.
\end{itemize}

\section*{Cap table}
\begin{itemize}[leftmargin=*]
  \item 100\% founder-owned.
  \item Seed round to offer 15–20\% equity to strategic capital partners.
\end{itemize}

\section*{Appendix: scientific appendix and data room}
\begin{itemize}[leftmargin=*]
  \item White papers, application notes, and benchmarks for each product.
  \item Code repositories, API documentation, and datasets.
  \item Clinical and research use summaries (anonymised).
  \item Available upon request or under NDA in shared data room.
\end{itemize}

\bibliographystyle{unsrtnat}
\bibliography{references}

\section*{Acronyms}
\begin{acronym}
  \acro{ARR}[ARR]{Annual Recurring Revenue}
  \acro{CAC}[CAC]{Customer Acquisition Cost}
  \acro{LTV}[LTV]{Lifetime Value}
  \acro{SaaS}[SaaS]{Software as a Service}
\end{acronym}

\end{document}





