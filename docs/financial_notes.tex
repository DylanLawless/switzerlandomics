\usepackage{geometry}
\geometry{
    a4paper,
    left=30mm,
    right=30mm,
    top=30mm,
    bottom=30mm,
}

\usepackage{placeins}
\usepackage[footnotes,definitionLists,hashEnumerators,smartEllipses,hybrid]{markdown}
\usepackage[utf8]{inputenc}
\usepackage[document]{ragged2e} % remove line justify
\usepackage{dblfloatfix}

\usepackage{microtype}
\usepackage[none]{hyphenat}

\setlength{\RaggedRightParindent}{1.5em}
\setlength{\parskip}{0.6em}

\usepackage{setspace}
\setstretch{1.2} % Optionally adjust line spacing

\setlength{\RaggedRightParindent}{\parindent}
\setlength{\parskip}{1em}

  
\usepackage{fontspec}
%\defaultfontfeatures{Mapping=tex-text,Scale=MatchLowercase}
\setmainfont{Source Sans Pro--Light}[BoldFont={Source Sans Pro-Regular}, ItalicFont={Source Sans Pro-Light Italic},]

\usepackage{xurl} % for '\url' macro
% \usepackage{xcolor}
\usepackage[usenames, dvipsnames, svgnames, table, x11names]{xcolor} % Support for a broad range of colors and additional color models
\definecolor{swissblack}{HTML}{333333}
\definecolor{swissdarkblue}{HTML}{023047}
\definecolor{swissdarkgreen}{HTML}{006666}
\definecolor{swissyellow}{HTML}{FFDC49}
\definecolor{swissred}{HTML}{ff5353}
\definecolor{swisslightblue}{HTML}{53c7ff}
\definecolor{swisslightgreen}{HTML}{50d079}
\definecolor{swisslightorange}{HTML}{ffac67}
\definecolor{darkred}{HTML}{cd0000}

\definecolor{swisslink}{HTML}{005595}%219EBC
% \color{swiss_black} %default

% command to use these colors and formatting; xspace for correct spacing including with punctuation marks.
\usepackage{xspace}
\newcommand{\variablesdarkgreen}[1]{\textbf{\textcolor{swissdarkgreen}{#1}}\xspace}
\newcommand{\variablesdarkred}[1]{\textbf{\textcolor{darkred}{#1}}\xspace}

% \color{swissblack} 

 % green #219EBC
% darkgreen #006666

\usepackage{authblk}
\usepackage{lineno}
\usepackage{enumitem}
\setlist[itemize]{noitemsep}

\usepackage{siunitx} % SI units


\usepackage[colorlinks]{hyperref}
\hypersetup{
    linkcolor={swisslink},
    citecolor={swisslink},
    filecolor=blue!50!black,
    urlcolor=swisslink,
}

\usepackage{natbib}
\setcitestyle{square,numbers,sort&compress}
\usepackage{hypernat}

\usepackage{graphicx}
\graphicspath{ {images/} }

\usepackage{booktabs}
\usepackage{rotating, tabularx}
\usepackage{ltablex}
\usepackage{caption}
\captionsetup{font=normalsize}

\usepackage{pdflscape}
\usepackage{multirow}

\usepackage{fancyhdr}
\pagestyle{fancy}
%\lhead{Lecture 1}
%\rhead{Handout 1}
%\lfoot{Lecture 1}
%\rfoot{Handout 1}
\usepackage{tocloft}  % Customizing the Table of Contents
\setcounter{tocdepth}{1}
\usepackage{pdfpages} % include supplemental pdf
% cover

%%%% Supplemental labels%%%%
%Define command to start a supplemental section
%set the supplemental letter used for figures (e.g. Figure E1)
\newcommand{\beginsupplement}{%
        \setcounter{table}{0}
        \renewcommand{\thetable}{E\arabic{table}}%
        \setcounter{figure}{0}
        \renewcommand{\thefigure}{E\arabic{figure}}%
         }

%\usepackage[a4paper,margin=5pt]{geometry}
% Define a new column type for left-aligned text in a fixed width column
% \newcolumntype{L}[1]{>{\raggedright\arraybackslash}p{#1}}
\newcolumntype{L}[1]{>{\noindent\arraybackslash\hspace{-1.5em}}p{#1}}


\begin{document}
\title{Key Financial Metrics for Early-Stage Genomics Companies}

\newcommand{\IPSNEO}{1}
\author[\IPSNEO]{Dylan Lawless %\thanks{Addresses for correspondence: \href{mailto:Dylan.Lawless@kispi.uzh.ch}{Dylan.Lawless@kispi.uzh.ch}}
}
\affil[\IPSNEO]{Switzerland Omics, Zürich, Switzerland.}

\maketitle
\justify

\section*{Acronyms}
\begin{acronym}
  \acro{ARR}[ARR]{Annual Recurring Revenue}
  \acro{CAC}[CAC]{Customer Acquisition Cost}
  \acro{LTV}[LTV]{Lifetime Value}
  \acro{SaaS}[SaaS]{Software as a Service}
\end{acronym}

\section*{Annual recurring revenue (\ac{ARR})}

\ac{ARR} is the total value of predictable, recurring revenue in a 12-month period. It is a key metric for evaluating the scalability and stability of businesses offering subscription or licensing-based services. Especially relevant for \ac{SaaS} or data-access models.

\textbf{Formula:}
\[
\text{\ac{ARR}} = \text{Monthly Recurring Revenue per Customer} \times \text{Number of Customers} \times 12
\]

\textbf{Example:}  
If 10 hospital labs are expected to pay CHF 12,000/year:
\[
\text{\ac{ARR}} = \text{CHF }12{,}000 \times 10 = \text{CHF }120{,}000
\]

\section*{Burn rate}

Burn rate refers to how much money a company spends each month, typically before generating revenue. It reflects your monthly net loss.

\textbf{Formula:}
\[
\text{Burn Rate} = \text{Monthly Expenses} - \text{Monthly Revenue}
\]

\textbf{Example:}  
If monthly expenses are CHF 10K and revenue is CHF 0:
\[
\text{Burn Rate} = \text{CHF }10{,}000
\]

\section*{Runway}

Runway is the amount of time (in months) a startup can continue operating at the current burn rate before exhausting its funds.

\textbf{Formula:}
\[
\text{Runway (months)} = \frac{\text{Cash on Hand}}{\text{Monthly Burn}}
\]

\textbf{Example:}  
With CHF 100K in cash and CHF 10K monthly burn:
\[
\text{Runway} = 10 \text{ months}
\]

\section*{Customer acquisition cost (\ac{CAC})}

\ac{CAC} measures the average cost to acquire one paying customer, factoring in marketing, sales, and onboarding costs.

\textbf{Formula:}
\[
\text{\ac{CAC}} = \frac{\text{Total Sales \& Marketing Costs}}{\text{Number of New Customers Acquired}}
\]

\section*{Lifetime value (\ac{LTV})}

\ac{LTV} estimates the total revenue a customer is expected to generate over their lifetime with your product or service.

\textbf{Formula:}
\[
\text{\ac{LTV}} = \text{Annual Revenue per Customer} \times \text{Average Retention (in years)}
\]

\section*{Gross margin}

Gross margin reflects how much profit remains after accounting for the cost of delivering the service. For data and software businesses, margins are typically high.

\textbf{Formula:}
\[
\text{Gross Margin} = \frac{\text{Revenue} - \text{Cost of Goods Sold}}{\text{Revenue}}
\]

\section*{Focus areas while pre-revenue}

\begin{itemize}[leftmargin=*]
  \item Know your burn rate and runway precisely.
  \item Define a clear pricing model (per report, per hospital, annual licence).
  \item Model a realistic \ac{ARR} path based on pilots and current usage.
  \item Prepare a 12–24 month financial plan outlining projected costs (hires, cloud infrastructure, compliance, legal, sales).
\end{itemize}


\bibliographystyle{unsrtnat}
\bibliography{references}

\end{document}





